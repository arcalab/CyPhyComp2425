\documentclass{beamer}
\usepackage{etex} % fixes new-dimension error
\usepackage{lmodern}
\usepackage[T1]{fontenc}
\usepackage[absolute,overlay]{textpos} % this is for textblock
\usepackage{tikz}
\usetikzlibrary{arrows.meta, calc, fit, tikzmark}
\usepackage{pgfplots}

%-------------- template --------------------------------------------------
\usetheme{metropolis}
\metroset{block=fill}
%\usetheme{Boadilla}

% Configuring the foot line
\setbeamertemplate{footline}
{
  \leavevmode%
  \hbox{%
  \begin{beamercolorbox}[wd=.4\paperwidth,ht=2.25ex,dp=1ex,center]{author in head/foot}%
    \usebeamerfont{author in head/foot}\insertshortauthor
  \end{beamercolorbox}%
  \begin{beamercolorbox}[wd=.5\paperwidth,ht=2.25ex,dp=1ex,center]{title in head/foot}%
    \usebeamerfont{title in head/foot}\insertsection
  \end{beamercolorbox}%
  \begin{beamercolorbox}[wd=.1\paperwidth,ht=2.25ex,dp=1ex,right]{date in head/foot}%
    \insertframenumber{} / \inserttotalframenumber\hspace*{2ex} 
  \end{beamercolorbox}}%
  \vskip0pt%
}
% No configuration symbols
\setbeamertemplate{navigation symbols}{}
%----------------------------------------------------------------------------

% context
\AtBeginSection[]
{
    \begin{frame}
        \frametitle{Table of Contents}
        \tableofcontents[currentsection]
    \end{frame}
}
\author[Renato Neves]{Renato Neves}
% logos of institutions
\titlegraphic{
  \begin{textblock*}{5cm}(6.7cm,7.57cm)
     \includegraphics[scale=0.05525]{images/uminho.png}
  \end{textblock*}
  \begin{textblock*}{5cm}(9.4cm,7.57cm)
    \includegraphics[scale=0.50]{images/haslab.pdf}
  \end{textblock*}
}
% No date
\date{}

\input{macros/preamble}
\input{macros/macros}

%----------------------------------------------------------------------------
\usepackage{graphicx,amsmath}
\usepackage{stmaryrd} % cf. interleave
\usepackage{booktabs}
\usepackage{amscd}

\usepackage{alltt}
%------ using xy ------------------------------------------------------------
\usepackage[all]{xy}
%\def\larrow#1#2#3{\xymatrix{ #3 & #1 \ar[l] _-{#2} }}
\def\larrow#1#2#3{\xymatrix{ #3 & #1 \ar[l] _--{#2} }}
\def\rarrow#1#2#3{\xymatrix{ #1 \ar[r]^-{#2} & #3 }}
\def\arLaw#1#2#3#4#5{
\xymatrix{
        #1      \ar@/^1pc/[rr]^-{#4} &
        #5 &
        #2      \ar@/^1pc/[ll]^-{#3}
}}
\def\arLeq#1#2#3#4{\arLaw{#1}{#2}{#3}{#4}\leq}
%------ using pstricks (rnode etc) ------------------------------------------
\usepackage{pstricks,pst-node,pst-text,pst-3d}
%------ using color ---------------------------------------------------------

\newrgbcolor{goldenrod}{.80392 .60784 .11373}
\newrgbcolor{darkgoldenrod}{.5451 .39608 .03137}
\newrgbcolor{brown}{.15 .15 .15}
\newrgbcolor{darkolivegreen}{.33333 .41961 .18431}
%
%
\def\gold#1{{\goldenrod #1}}
% \def\dgold#1{{\darkgoldenrod #1}}
\def\dgold#1{{\alert{#1}}}
%\def\brw#1{{\brown #1}}
\def\dkb#1{{\blue #1}}
\def\tdkb#1{\textbf{\darkblue #1}}
%%\def\gre#1{{\green #1}}
\def\gre#1{{\darkolivegreen #1}}
\def\gry#1{{\gray #1}}
\def\rdb#1{{\red #1}}
\def\st{\mathbf{.}\,}
\def\laplace#1#2{*\txt{\mbox{ \fcolorbox{black}{myGray}{$\begin{array}{c}\mbox{#1}\\\\#2\\\\\end{array}$} }}}
%\newcommand{\galois}[2]{#1\; \dashv\; #2}

\def\eqm{\mathbin{\equiv}}                     
\def\noeqm{\mathbin{\not\!\equiv}}  
%\newcommand{\flam}[2]{\lambda_{#1}\; .\; #2}
\def\existential#1#2{\exists_{#1}\;.\; #2}
\def\existencial#1#2{\exists_{#1}\;.\; #2}

\def\pv#1#2{\langle #1 \rangle #2}
\def\nc#1#2{[#1]#2}
\def\pvo#1#2{\langle \! \! \! \langle #1 \rangle \! \! \! \rangle\, #2}
\def\nco#1#2{\llbracket #1 \rrbracket #2}
\def\cvg#1{\llbracket \downarrow \rrbracket #1}
\def\cvgr#1#2{\llbracket #1 \downarrow \rrbracket #2}
\def\cvgl#1#2{\llbracket \downarrow  #1 \rrbracket #2}
\def\cvglr#1#2{\llbracket \downarrow  #1 \downarrow \rrbracket #2}
\def\lfp#1#2{\mu {#1}\, .\, {#2}}
\def\lpf#1#2{\mu {#1}\, .\, {#2}}
\def\gfp#1#2{\nu {#1}\, .\, {#2}}
\def\gpf#1#2{\nu {#1}\, .\, {#2}}
\def\mset#1{\vvv #1 \vvv}
\def\vvv{\vert \! \vert}
\def\mnc#1{\vvv [#1] \vvv}
\def\mpv#1{\vvv \langle #1 \rangle \vvv}
\def\bcomp#1{#1^{\text{c}}}
\def\eqm{\mathbin{\simeq}}
\def\noeqm{\mathbin{\not\!\simeq}}
\def\universal#1#2{\forall_{#1}\;.\; #2}
\def\existential#1#2{\exists_{#1}\;.\; #2}
\def\oexistential#1#2{\exists^{1}_{#1}\;.\; #2}
\def\MM{\mathcal{M}}
\def\uppaal{\textsc{Uppaal}}
\def\cc#1{\mathcal{C}(#1)}
\def\R{\mathcal{R}}
\def\TL#1{\mathcal{T}(#1)}
\def\ET#1{\mathsf{ExecTime(#1)}}


\begin{document}

\title{Timed Automata}

\frame[plain]{\titlepage}

\begin{frame}{Last lecture}
  Visited \alert{\underline{syntax}} and \alert{\underline{semantics}}

  Analysed central ideas in \alert{\underline{concurrency}} and
  \alert{\underline{synchronisation}}

  \pause
  \vfill
  We will now see how \alert{\underline{time}} fits in the items above
\end{frame}
\section{Motivation}
%----------------------------------------------------------------------------------
\begin{slide}{Motivation}


Saying that an airbag in a car crash \alert{eventually
inflates} is insufficient -- it would be better to say that \dots
\begin{center}
\dots\ in a car crash the airbag inflates \alert{\underline{within 20ms}}
\end{center}
 
\vfill
\emph{Correctness in time-critical systems not only depends on
      the logical result of the computation, but also \alert{on \underline{the time}
      at which the results are produced}}
\begin{flushright}
[Baier \& Katoen, 2008]
\end{flushright}
\end{slide}

\begin{slide}{Examples of time-critical systems}

\begin{block}{(Network-based) traffic lights}
  Lights activate at specific time intervals
\end{block}

\smallskip
\begin{block}{(Re)transmission  protocols}
  Communication of large files between a remote unit and a video/audio
  equipment. 
\end{block}

\smallskip
\begin{block}{Many others}
Pacemakers, autonomous driving, electric grids
\dots
\end{block}
\end{slide}

\begin{slide}{This chapter}
        We will explore an \alert{\underline{automaton-based formalism}} with an explicit notion of 
        a \alert{\underline{clock}} 
\begin{center}
\fbox{Timed Automata [Alur \& Dill, 90]}
\end{center}


\bigskip
Associated tool
\begin{itemize}
\item \uppaal\ [Behrmann, David, Larsen, 04]
\end{itemize}
\end{slide}
%----------------------------------------------------------------------------------
\begin{slide}{\uppaal}

\begin{center}
\fbox{\uppaal\ = (\alert{Upp}sala University + \alert{Aal}borg University) [1995]}
\end{center}

\begin{itemize}
\item Toolbox for modelling and analysis of timed systems
\item Systems modelled as \alert{\underline{networks}} of timed automata 
        with \alert{\underline{channel synchronisations}}
\item Properties specified in a \alert{\underline{temporal logic}} 
\end{itemize}

\end{slide}

\section{The very basics of timed automata}


\begin{slide}{Timed automata}
Finite-state machines equipped with \alert{\underline{real-valued clocks}}

\begin{itemize}
\item clocks can only be read or
\item reset to zero (after which they start increasing their value again as time progresses)
\item a clock's value corresponds to time elapsed since its last reset 
\item all clocks proceed synchronously (\emph{i.e.} at the same rate)
\end{itemize}
\end{slide}
% % ----------------------------------------------------------------------------------

\begin{frame}{Example: the annoying lamp}
  
\begin{figure}[htb]
  \centering
  \includegraphics[scale=0.35]{./images/Lamp.pdf}\\
\end{figure}

\centering
{\scriptsize (extracted from \uppaal)}

\end{frame}

\begin{slide}{Guards, updates, and invariants}
\small \centering

\begin{tabular}{c}
   \includegraphics[width=8cm]{./images/model0.jpg} \\  \includegraphics[width=7cm]{./images/model1.jpg}
\end{tabular}

\end{slide}

\begin{slide}{Timed automata}
\small
A timed automaton is a tuple
$\pair{\mathbf{L}, \mathbf{L_0}, \mathbf{Act}, \mathbf{C}, \mathbf{Tr}, 
\mathbf{Inv}}$ 
\begin{itemize}
     \item $\mathbf{L}$ a set of locations and 
             $\mathbf{L_0} \subseteq \mathbf{L}$ set
        of \alert{initial} locations
\item $\mathbf{Act}$ set of actions (\alert{channels}) 
        and $\mathbf{C}$ set of clocks
     \item
             $\mathbf{Tr} \subseteq \mathbf{L} 
             \times \cc{\mathbf{C}} \times \mathbf{Act} 
             \times \pow{(\mathbf{C})} \times \mathbf{L}$
     is a \alert{transition} relation
\begin{equation*}
  \ell_1\; \tran{g,a,U}\;  \ell_2
\end{equation*}
transition from location $\ell_1$ to $\ell_2$, labelled by $a$, enabled if
\alert{guard} $g$ holds; when performed resets the set $U$ of clocks
\item $\fdec{\mathbf{Inv}}{\mathbf{L}}{\cc{\mathbf{C}}}$ assignment of invariants to
  locations
\end{itemize}
$\cc{\mathbf{C}}$ denotes the set of clock constraints over a set $C$ of
clocks
\end{slide}

\begin{slide}{Clock constraints}

Each constraint is formed according to 
\begin{equation*}
  g \; ::=\; x \mathbin{\square} n \; \mid\; x - y \mathbin{\square} n  \; \mid\; g \e g \;\mid\; true
\end{equation*}
where $x,y \in C, n \in \N$ and $\square \in \enset{<,\leq,>,\geq,=}$

This is used in
\begin{itemize}
\item transitions (enabling conditions) -- a transition cannot occur
  if its guard is false
\item locations (safety conditions) -- a location must be left before
  its invariant becomes false
\end{itemize}

\end{slide}

\begin{frame}{A revisit of the annoying lamp}
  
\begin{figure}[htb]
  \centering
  \includegraphics[scale=0.35]{./images/Lamp.pdf}\\
\end{figure}

Exercise: define $\pair{L, L_0, Act, C, Tr, Inv}$ 

\end{frame}

\section{Parallel Composition}

\begin{frame}{Parallel composition of timed automata}

        Action labels as channels

        Communication mechanism analogous to CCS

        Shared clocks
\end{frame}

\begin{slide}{Definition}

  Let $H = Act_1 \cap Act_2$. The parallel composition of
  $ta_1$ and $ta_2$ synchronising on $H$ is the timed automaton

  \vspace{0.2cm}
$ta_1 \parallel_H ta_2 := \pair{L_1 \times L_2, L_{0,1} \times L_{0,2}, 
\mathbf{Act}, C_1 \cup C_2, \mathbf{Tr}, 
\mathbf{Inv}}$

  \vspace{0.2cm}
\begin{itemize}
\item $\mathbf{Act} = ((Act_1 \cup Act_2) - H) \cup \enset{\tau}$
\item $\mathbf{Inv} (\ell_1,\ell_2) = Inv_1(\ell_1) \e  Inv_2(\ell_2)$
\item $\mathbf{Tr}$ is given by:
\begin{itemize}
\item $(\ell_1,\ell_2) \tran{g,a,U} (\ell'_1,\ell_2)\; $ if $\; a \not \in H \e  \ell_1 \tran{g,a,U} \ell'_1 $
\item $(\ell_1,\ell_2) \tran{g,a,U} (\ell_1,\ell'_2)\; $ if $\; a \not \in H \e   \ell_2 \tran{g,a,U} \ell'_2$
\item $(\ell_1,\ell_2) \tran{g,\tau,U} (\ell'_1,\ell'_2)\; $ if $\; a \in H \e  \ell_1 \tran{g_1,a,U_1} \ell'_1 \e \ell_2 \tran{g_2,\overline{a},U_2} \ell'_2$\\
with $g = g_1 \e g_2$ and $U = U_1 \cup U_2$
\end{itemize}
\end{itemize}
\end{slide}

\begin{slide}{Example: (re)revisiting the lamp interrupt}
\small

\begin{figure}[htb]
  \centering
  \includegraphics[scale=0.25]{./images/Lamp.pdf}
  \includegraphics[scale=0.09]{./images/User.pdf}\\
\end{figure}


\vfill
\begin{block}{In \uppaal}
\begin{itemize}
\item Complementary
  actions marked by \alert{?} and \alert{!} annotations
\end{itemize}
\end{block}
\end{slide}

% %----------------------------------------------------------------------------------
\begin{slide}{Exercise: worker, hammer, nail}
        Write down the parallel composition of the following automata

\begin{figure}[htb]
  % Requires \usepackage{graphicx}
  \includegraphics[width=50mm]{./images/Worker.pdf}
  \hspace{0.5cm}
  \includegraphics[width=50mm]{./images/Hammer.pdf}

  \vspace{0.5cm}
  \includegraphics[width=50mm]{./images/Nail.pdf}
\end{figure}
\end{slide}
%%%%%%%%%%%%%%
\section{Semantics}
%%%%%%%%%%%%
% %----------------------------------------------------------------------------------
\begin{slide}{Semantics of timed automata}
\small\centering
\begin{tabular}{lc@{~~}l}
\toprule 
\dgold{Syntax} && \dgold{Semantics}\\
\cmidrule(lr){1-3}
% Process Languages (eg CCS) &  & LTS (Labelled Transition Systems)\\
\gry{\emph{How to write}} & & \gry{\emph{How to execute}}\\
CCS &  & LTS \\
Timed Automaton &  & TLTS (Timed LTS) \\
\bottomrule
\end{tabular}
~\\
~\\
~\\

\vfill
\begin{block}{Timed LTS}
Introduce \alert{delay transitions} to capture the passage of time
\begin{align*}
s \tran{a} s' & \;\; \text{for $a \in Act$, are ordinary transitions due to action occurrence }\\
s \tran{d} s' &\; \;  \text{for $d \in \mathbb{R}_{\geq 0}$, are \alert{delay} transitions}
\end{align*}
subject to sanity constraints \dots 
\end{block}
\end{slide}
% %----------------------------------------------------------------------------------
\begin{slide}{Timed LTS pt. II}
        \begin{enumerate}
                \item
        Time additivity
        \begin{equation*}
        (s \tran{d} s'  \e 0 \leq d' \leq d)\; \imp\; s  \tran{d'} s'' \tran{\; d-d'}           s' \; \text{for some state $s''$}
        \end{equation*}
        \item Delay transitions are deterministic
        \begin{equation*}
        (s \tran{d} s'  \e s \tran{d} s'')\; \imp\; s'  = s''
        \end{equation*}
        \end{enumerate}
\end{slide}
% %----------------------------------------------------------------------------------
\begin{slide}{The semantics}
Every TA $ta$ defines a TLTS 
\begin{equation*}
\TL{ta}
\end{equation*}
whose states are pairs 
\begin{equation*}
\pair{\text{location}, \text{clocks valuations}}
\end{equation*}
\end{slide}
% %----------------------------------------------------------------------------------
\begin{slide}{Clock valuation}

\begin{block}{Definition}
A clock valuation \alert{$\eta$} for a set of clocks $C$ is a function 
\begin{equation*}
\fdec{\alert{\eta}}{C}{\mathbb{R}_{\geq 0}}
\end{equation*}
assigning to each clock $x \in C$ its current value $\alert{\eta}\, x$
\end{block}
~\\

\begin{block}{Satisfaction of Clock Constraints}
\begin{align*}
\alert{\eta} \models x \mathbin{\square} n \; & \dimp\; \alert{\eta}\, x \mathbin{\square} n\\
\alert{\eta} \models x-y \mathbin{\square} n \; & \dimp\; (\alert{\eta}\, x - \alert{\eta}\, y) \mathbin{\square} n\\
\alert{\eta} \models g_1 \e g_2 \; & \dimp\; \alert{\eta} \models g_1 \e \alert{\eta} \models g_2
\end{align*}
\end{block}
\end{slide}

\begin{slide}{Operations on clock valuations}
\begin{block}{Delay}
For each $d \in \mathbb{R}_{\geq_0}$, valuation $\eta \alert{+ d}$ is given by
\begin{equation*}
(\eta \alert{+ d})\, x\; =\; \eta\, x\; +\; d
\end{equation*}
\end{block}
~\\

\begin{block}{Reset}
For each $R \subseteq C$, valuation $\eta\alert{[R]}$ is given by
\begin{equation*}
\begin{cases}
\eta\alert{[R]}\, x\; =\; \eta\, x & \; \text{if}\; x \not\in R\\
\eta\alert{[R]}\, x\; =\; 0 & \; \text{if}\; x \in R
\end{cases}
\end{equation*}
\end{block}
\end{slide}

\begin{slide}{From $ta$ to $\TL{ta}$}
\small
Let $ta = \pair{L, L_0, Act, C, Tr, Inv}$
\begin{equation*}
        \TL{ta}\; =\; \pair{\mathbf{S}, \mathbf{S_0} \subseteq S, \mathbf{N}, 
        \mathbf{T}}
\end{equation*}
where
\begin{itemize}
\item
        $\mathbf{S} = \setdef{(l,\eta) \in L \times
    (\mathbb{R}_{\geq 0})^C}{\eta \models Inv(l)}$
\item
        $\mathbf{S_0} = \setdef{(\ell_0,\eta)}{\ell_0 \in L_0\; \e\;
    \eta\, x = 0\; \text{for all $x \in C$}}$
\item $\textbf{N} = Act + \mathbb{R}_{\geq 0}$ (i.e., transitions can be
    labelled by actions or delays)
\item $\textbf{T} \subseteq S \times N \times S$ is given by
\end{itemize}
\begin{align*}
(l,\eta) \tran{a} (l',\eta')\; & ~~\text{if}~~\; 
\exists_{l \tran{g,a,U} l' \in Tr}  \; \; \eta \models g \; \e\; \eta' = \eta[U] \; \e\; \eta' \models Inv(l')\\
(l,\eta) \tran{d} (l,\eta +d)\; & ~~\text{if}~~\; 
\eta + d \models Inv(l)
\end{align*}
\end{slide}
\begin{slide}{Example: the simple switch pt. I}
\small
\begin{figure}[htb]
  \centering
  \includegraphics[width=5cm]{./images/SwitchA.pdf}\\
\end{figure}

\vfill
\begin{block}{$\TL{\textsf{Simple switch}}$}
\begin{equation*}
S = \setdef{(off,t)}{t \in \mathbb{R}_{\geq 0}} \cup \setdef{(on,t)}{0 \leq t \leq 2} 
\end{equation*}
where $t$ is a shorthand for $\eta$ such that $\eta\, x = t$
\end{block}
\end{slide}
% % ----------------------------------------------------------------------------------
\begin{slide}{Example: the simple switch pt. II}

\begin{figure}[htb]
  \centering
  \includegraphics[width=4cm]{./images/SwitchA.pdf}\\
\end{figure}
\vfill
\begin{block}{$\TL{\textsf{Simple switch}}$}
\begin{align*}
(off,t) \tran{d} (off,t+d)\; & \; \text{for all $t,d \geq 0$}\\
(off,t) \tran{in?} (on,0)\; & \; \text{for all $t \geq 0$}\\
(on,t) \tran{d} (on,t+d)\; & \; \text{for all $t,d \geq 0$ and $t+d \leq 2$}\\
(on,t) \ \tran{out!} (off,t)\; & \; \text{for all $1 \leq t \leq 2$}
\end{align*}
\end{block}
\end{slide}

\end{document}
