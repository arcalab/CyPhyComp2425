\documentclass{beamer}
\usepackage{etex} % fixes new-dimension error
\usepackage{lmodern}
\usepackage[T1]{fontenc}
%-------------- template --------------------------------------------------
\usetheme{metropolis}
\metroset{block=fill}
%\usetheme{Boadilla}

% Configuring the foot line
\setbeamertemplate{footline}
{
  \leavevmode%
  \hbox{%
  \begin{beamercolorbox}[wd=.4\paperwidth,ht=2.25ex,dp=1ex,center]{author in head/foot}%
    \usebeamerfont{author in head/foot}\insertshortauthor
  \end{beamercolorbox}%
  \begin{beamercolorbox}[wd=.5\paperwidth,ht=2.25ex,dp=1ex,center]{title in head/foot}%
    \usebeamerfont{title in head/foot}\insertsection
  \end{beamercolorbox}%
  \begin{beamercolorbox}[wd=.1\paperwidth,ht=2.25ex,dp=1ex,right]{date in head/foot}%
    \insertframenumber{} / \inserttotalframenumber\hspace*{2ex} 
  \end{beamercolorbox}}%
  \vskip0pt%
}
% No configuration symbols
\setbeamertemplate{navigation symbols}{}
%----------------------------------------------------------------------------
\usepackage{graphicx,amsmath}
\usepackage{stmaryrd} % cf. interleave
\usepackage{booktabs}
\usepackage{amscd}
\usepackage{multicol}
\usepackage[absolute,overlay]{textpos}
\usepackage{alltt}
\usepackage{proof}
\usepackage{subcaption}
\usepackage[all]{xy}
\usepackage{tikz}
\usetikzlibrary{arrows.meta, calc, fit, tikzmark}
\usepackage{pgfplots}
\usepackage{listings}
% \usepackage{textcomp}
% \usepackage{tikz-cd}
% \usepackage[new]{old-arrows}
% \usepackage[all]{xy}

%------ using pstricks (rnode etc) ------------------------------------------
\usepackage{pstricks,pst-node,pst-text,pst-3d}
% ------ using color ---------------------------------------------------------
\newrgbcolor{goldenrod}{.80392 .60784 .11373}
\newrgbcolor{darkgoldenrod}{.5451 .39608 .03137}
\newrgbcolor{brown}{.15 .15 .15}
\newrgbcolor{darkolivegreen}{.33333 .41961 .18431}
\def\gold#1{{\goldenrod #1}}
\def\dgold#1{{\alert{#1}}}
\def\dkb#1{{\blue #1}}
\def\tdkb#1{\textbf{\darkblue #1}}
\def\gre#1{{\darkolivegreen #1}}
\def\gry#1{{\gray #1}}
\def\rdb#1{{\red #1}}
% ----------------------------------------------------------------------------

% context
\AtBeginSection[]
{
    \begin{frame}
        \frametitle{Table of Contents}
        \tableofcontents[currentsection]
    \end{frame}
}
\author[Renato Neves]{Renato Neves}
% logos of institutions
\titlegraphic{
  \begin{textblock*}{5cm}(6.7cm,7.59cm)
     \includegraphics[scale=0.054]{images/uminho.png}\hspace*{.85cm}~%
  \end{textblock*}
  \begin{textblock*}{5cm}(9.4cm,7.57cm)
    \includegraphics[scale=0.50]{images/haslab.pdf}
  \end{textblock*}
}
% No date
\date{}

\input{macros}
\begin{document}

\title{Calculus of Communicating Systems}

\frame[plain]{\titlepage}

\section{Syntax and Semantics of Programming}

\begin{frame}{A sprinkle of linguistics}
  We will often face two linguistic concepts that every programmer
  ought to know
  \begin{itemize}
  \item syntax - determines whether a sentence
    is valid or not
  \item semantics - the meaning of valid sentences
  \end{itemize}

  \vfill
  \begin{example}[syntax]
      The sentence (program) $\mathtt{x := p\, ;q}$ is forbidden by
      the syntactic rules of most programming languages
    \end{example}
  \begin{example}[semantics]
      The sentence (program) $\mathtt{x := 1}$ has the meaning ``writes
      \texttt{1} in the memory address corresponding to \texttt{x}''
  \end{example}
\end{frame}

\begin{frame}{The need for semantics in programming}

  How can one prove that a program does what is supposed to do if its
  semantics (i.e. its meaning) is not established \emph{a priori}?

  \vfill
  \begin{examples}
    What are the outputs of the following programs?
    \begin{itemize}
            \item $f()\{ \mathtt{print}\ a; \mathtt{ret.\ 1} \} ;\
                g()\{ \mathtt{print}\ b; \mathtt{ret.\ 0} \};\
                x:= f() + g()$ 
            \item $x:=2\ ; (x := x + 1 \parallel  x := 0)$ 
            \item $(a:= 1\, ; \mathtt{print}\ b) \parallel 
                    (b:= 1\, ; \mathtt{print}\ a)$ 
    \end{itemize}
  \end{examples}

\end{frame}

\begin{frame}{Transition systems as semantic providers}

  Transition systems are an \alert{ubiquitous mechanism} for defining the
  semantics of programming languages 

  \vfill
  Following tradition, we will use them to define the
  semantics of a simple (but powerful !!) concurrent language ---

  and then base on this learning step to tackle Dijkstra's 

  \vspace{3.5pt}
  \begin{minipage}[0.3\textheight]{\textwidth}
  \begin{columns}[c]
  \begin{column}{0.7\textwidth}
    Dining Philosophers Problem (\emph{circa} 1965)
  \end{column}
  \begin{column}{0.22\textwidth}
    \includegraphics[scale=1.1]{images/Dijkstra.jpg}
  \end{column}
  \end{columns}
  \end{minipage}

\end{frame}

\section{A Simple Concurrent Language and its Semantics}

\begin{frame}{Calculus of Communicating Systems}

  \begin{minipage}[0.3\textheight]{\textwidth}
  \begin{columns}[c]
  \begin{column}{0.8\textwidth}
    \begin{block}{Syntax}
      $P,Q ::= X \mid a . P\ |\ \sum_{ i \in I} P_i \ \mid P \parallel Q 
      \mid P \backslash L \mid \dots$
    \end{block}
  \end{column}
  \begin{column}{0.23\textwidth}
    \includegraphics[scale=0.25]{images/Milner.jpg}
  \end{column}
  \end{columns}
  \end{minipage}
  \begin{itemize}
  \item $X$ is a process name
  \item $a. P$ \underline{communicates} via \underline{channel} $a$ and proceeds as $P$
  \item $\sum_{ i \in I} P_i$ non-deterministic choice between processes $P_i$
  \item $P \parallel Q$ \underline{parallel composition} 
          between processes $P$ and $Q$
  \item $P \backslash L$ makes channels in $L$ private `outside' of $P$
  \end{itemize}
\end{frame}

\begin{frame}{First Steps with CCS}
        \begin{block}{Conventions}
                 \begin{itemize}
                  \item $\mathtt{0} = \sum_{i \in \emptyset} P_i$ 
                          (denotes a terminating process)
                  \item $\bar{a}$ denotes outgoing information via channel $a$ 
                  \item $\tau$ denotes an invisible action
                  \end{itemize}
        \end{block}
  
        \begin{examples}[processes in CCS]
          \begin{itemize}
                \item $a . \mathtt{0} \parallel \bar{a} . \mathtt{0}$ - two processes
                        connected via channel $a$; information flows in one direction
                \item $a . \bar{b} . \mathtt{0} \parallel \bar{a} . b . \mathtt{0}$ -
                        info. flows in one direction via $a$ and then in the inverse
                        direction via $b$ 
                \item $(a . \bar{b} . \mathtt{0} \parallel \bar{a} . b . \mathtt{0})
                  \backslash \{a,b\}$ - both channels $a,b$ now private
          \end{itemize}
        \end{examples}
\end{frame}

\begin{frame}{First steps with CCS}

  Which of these expressions are valid sentences in CCS?

  \begin{enumerate}
  \item $a . b . P + Q$
  \item $a + b$
  \item $P . a$
  \item $(P + Q). a$
  \item $a . \mathtt{0} + b . \mathtt{0}$
  \item $P . Q$
  \end{enumerate}
\end{frame}

\begin{frame}{CCS and Cyclic Behaviour}
  We now add the construct \alert{$\mathtt{rec}\ X.\ P$} to the syntax of CCS -- so that we
  can describe cyclic behaviour

  \vfill
  \begin{example}
    $\mathtt{rec}\ X.\ a . b . X$  - receive communication through $a$ and then
    through $b$; after that repeat protocol
  \end{example}

  \begin{example}[the coffee machine and the student]
    $(\mathtt{rec}\ X.\ coin . \overline{coffee} . X)\ ||\ (\mathtt{rec}\ Y.\
    \overline{coin} .  coffee . \overline{wrk}. Y)$
  \end{example}

  \vfill
  Write down a coffee machine that fails to deliver coffee sometimes
\end{frame}

\begin{frame}{Semantics of CCS}
        Every process yields a transition system
        according to the rules 
  \begin{flalign*}
          \infer[(\textbf{pr})]{a. P \stackrel{a}{\longrightarrow} P}{} \hspace{0.5cm}
          \infer[(\textbf{ch})]{\sum_{i \in I} P_i \stackrel{a}{\longrightarrow} Q}{P_i \stackrel{a}
      {\longrightarrow} Q} \hspace{0.5cm}
      \infer[(\textbf{res}) a,\overline{a} \not \in L]{P \backslash L \stackrel{a}{\longrightarrow}
      P' \backslash L}{P \stackrel{a}{\longrightarrow} P'} 
  \end{flalign*}
  \begin{flalign*}
          \infer[(\textbf{coml})]{P \parallel Q \stackrel{a}{\longrightarrow} P' \parallel Q}{P
      \stackrel{a}{\longrightarrow} P'} \hspace{0.8cm}
      \infer[(\textbf{comr})]{P \parallel Q \stackrel{a}{\longrightarrow} P \parallel Q'}{Q
      \stackrel{a}{\longrightarrow} Q'} \hspace{0.8cm}
  \end{flalign*}
  \begin{flalign*}
          \infer[(\textbf{com})]{P \parallel Q \stackrel{\tau}{\longrightarrow} P' \parallel Q'}{P
      \stackrel{a}{\longrightarrow} P' \qquad Q \stackrel{\bar{a}}{\longrightarrow} Q'}
      \hspace{1cm}
      \infer[(\textbf{rec})]{\mathtt{rec}\ X.\ P \stackrel{a}{\longrightarrow} P'}{P \tikzmark{ss1}
      [ \tikzmark{ss2}\mathtt{rec}\ X.\ P / X] 
      \stackrel{a}{\longrightarrow} P' }
  \end{flalign*}

  \begin{tikzpicture}[overlay,remember picture,
     box/.style = {rounded corners},
     pin edge={-Stealth,thick, red}]
     \coordinate (ss1) at ($({pic cs:ss1})+(+0.5ex, 1.5ex)$);
     \coordinate (ss2) at ($({pic cs:ss2})+(-0.5ex,-0.5ex)$);
     \node[semitransparent, 
           fit=(ss1) (ss2),
           pin=below:\tiny{Substitution of $X$ in $P$ by $\mathtt{rec}\ X.\ P$}]  {};
  \end{tikzpicture}
\end{frame}

\begin{frame}{First steps with CCS}

  What are the semantics of the following processes?

  \begin{enumerate}
          \item $a . b . \mathtt{0}$
          \item $a. b. \mathtt{0} + c. d. \mathtt{0}$
          \item $a. b. \mathtt{0} \parallel c. d. \mathtt{0}$
          \item $\mathtt{rec}\ X.\ a. b. X$
  \end{enumerate}
\end{frame}

\section{Putting things into practice}

\begin{frame}{CCS at Work}
  \begin{minipage}[0.3\textheight]{\textwidth}
  \begin{columns}[c]
  \begin{column}{0.62\textwidth}
    With the syntax and semantics of CCS now in place, we put on
    our working hats and start to \alert{formally} analyse
    communication and synchronisation mechanisms
  \end{column}
  \begin{column}{0.28\textwidth}
    \includegraphics[scale=0.1]{images/mad-hatter.png}
  \end{column}
  \end{columns}
  \end{minipage}
\end{frame}


\begin{frame}{Starvation and Mutual Exclusion in CCS}
  We define three recursive processes
  \begin{flalign*}
    S & = \mathtt{rec}\, X.\ \overline{start} . finish . X & \qquad \quad (\text{the semaphore})\\
    P_1 & = \mathtt{rec}\, Y.\ start . a_1  . b_1. \overline{finish} . Y &\qquad (\text{process 1}) \\
    P_2 & = \mathtt{rec}\, Z.\ start . a_2  . b_2. \overline{finish} . Z &\qquad (\text{process 2})
  \end{flalign*}
  and then write down $(S \parallel P_1 \parallel P_2) \backslash \{ start, finish \}$

  Question: will we ever observe a sequence of actions $x_1 \dots x_n \dots$
  such that $x_i = \tikzmark{cr1} a_1 \tikzmark{cr2}$ and $x_{i + 1} = a_2$?
  \begin{tikzpicture}[overlay,remember picture,
     box/.style = {rounded corners},
     pin edge={-Stealth,thick, red}]
     \coordinate (ss1) at ($({pic cs:cr1})+(+0.5ex, 1.5ex)$);
     \coordinate (ss2) at ($({pic cs:cr2})+(-0.5ex,-0.5ex)$);
     \node[semitransparent, 
           fit=(ss1) (ss2),
           pin=below:\tiny{think of $a_i$ as writing on a critical region and of $b_i$
           as ending this process}]  {};
  \end{tikzpicture}

\end{frame}

\begin{frame}{Dining Philosophers Problem}
  Two philosophers sitting at the table in front of each other \dots thinking
  \dots

  They will wish to eat and for that effect there are precisely \alert{two
  forks} on the table, at their left and right-hand sides

  When Philosopher 1 wishes to eat he picks the fork on his left and then the
  one on his right

  Phil. 2 picks the fork on her left and then the fork on her right

  \vfill
  Write down this system in CCS and discover whether it is possible that both
  philosophers can no longer eat
\end{frame}

\begin{frame}{Going beyond the Dining Philosophers Problem \dots}
        Detection of both \alert{deadlocks} and \alert{livelocks} \eg\ in
        \begin{itemize}
                \item Driving systems 
                \item Pacemakers
                \item at the LHC
                \item \dots
        \end{itemize}

        See details at \url{https://www.mcrl2.org/web/index.html}
\end{frame}

\begin{frame}{\dots and beyond CCS}
        Different extensions of CCS to the 
        \begin{itemize}
                \item probabilistic
                \item quantum
                \item and \alert{\underline{timed}}
        \end{itemize}
        domains, among others
        

        \vfill
        Stay tuned!
\end{frame}
\nocite{milner80}
\bibliographystyle{amsalpha}
\bibliography{biblio.bib}

\end{document}
