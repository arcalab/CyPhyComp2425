\documentclass{beamer}
\usepackage{lmodern}
\usepackage[T1]{fontenc}
\usepackage[absolute,overlay]{textpos} % this is for textblock
\usepackage{tikz}
\usetikzlibrary{arrows.meta, calc, fit, tikzmark}
\usepackage{pgfplots}
\usepackage{hyperref}
%-------------- template --------------------------------------------------
\usetheme{metropolis}
\metroset{block=fill}
%\usetheme{Boadilla}

% Configuring the foot line
\setbeamertemplate{footline}
{
  \leavevmode%
  \hbox{%
  \begin{beamercolorbox}[wd=.4\paperwidth,ht=2.25ex,dp=1ex,center]{author in head/foot}%
    \usebeamerfont{author in head/foot}\insertshortauthor
  \end{beamercolorbox}%
  \begin{beamercolorbox}[wd=.5\paperwidth,ht=2.25ex,dp=1ex,center]{title in head/foot}%
    \usebeamerfont{title in head/foot}\insertsection
  \end{beamercolorbox}%
  \begin{beamercolorbox}[wd=.1\paperwidth,ht=2.25ex,dp=1ex,right]{date in head/foot}%
    \insertframenumber{} / \inserttotalframenumber\hspace*{2ex} 
  \end{beamercolorbox}}%
  \vskip0pt%
}
% No configuration symbols
\setbeamertemplate{navigation symbols}{}
%----------------------------------------------------------------------------

% context
\AtBeginSection[]
{
    \begin{frame}
        \frametitle{Table of Contents}
        \tableofcontents[currentsection]
    \end{frame}
}
\author[Renato Neves]{Renato Neves}
% logos of institutions
\titlegraphic{
  \begin{textblock*}{5cm}(6.7cm,7.57cm)
     \includegraphics[scale=0.05525]{./images/uminho.png}
  \end{textblock*}
  \begin{textblock*}{5cm}(9.4cm,7.57cm)
    \includegraphics[scale=0.50]{./images/haslab.pdf}
  \end{textblock*}
}
% No date
\date{}

\input{macros/preamble}
\input{macros/macros}

%----------------------------------------------------------------------------
\usepackage{graphicx,amsmath}
\usepackage{stmaryrd} % cf. interleave
\usepackage{booktabs}
\usepackage{amscd}

\usepackage{alltt}
%------ using xy ------------------------------------------------------------
\usepackage[all]{xy}
%\def\larrow#1#2#3{\xymatrix{ #3 & #1 \ar[l] _-{#2} }}
\def\larrow#1#2#3{\xymatrix{ #3 & #1 \ar[l] _--{#2} }}
\def\rarrow#1#2#3{\xymatrix{ #1 \ar[r]^-{#2} & #3 }}
\def\arLaw#1#2#3#4#5{
\xymatrix{
        #1      \ar@/^1pc/[rr]^-{#4} &
        #5 &
        #2      \ar@/^1pc/[ll]^-{#3}
}}
\def\arLeq#1#2#3#4{\arLaw{#1}{#2}{#3}{#4}\leq}
%------ using pstricks (rnode etc) ------------------------------------------
\usepackage{pstricks,pst-node,pst-text,pst-3d}
%------ using color ---------------------------------------------------------

\newrgbcolor{goldenrod}{.80392 .60784 .11373}
\newrgbcolor{darkgoldenrod}{.5451 .39608 .03137}
\newrgbcolor{brown}{.15 .15 .15}
\newrgbcolor{darkolivegreen}{.33333 .41961 .18431}
%
%
\def\gold#1{{\goldenrod #1}}
% \def\dgold#1{{\darkgoldenrod #1}}
\def\dgold#1{{\alert{#1}}}
%\def\brw#1{{\brown #1}}
\def\dkb#1{{\blue #1}}
\def\tdkb#1{\textbf{\darkblue #1}}
%%\def\gre#1{{\green #1}}
\def\gre#1{{\darkolivegreen #1}}
\def\gry#1{{\gray #1}}
\def\rdb#1{{\red #1}}
\def\st{\mathbf{.}\,}
\def\laplace#1#2{*\txt{\mbox{ \fcolorbox{black}{myGray}{$\begin{array}{c}\mbox{#1}\\\\#2\\\\\end{array}$} }}}
%\newcommand{\galois}[2]{#1\; \dashv\; #2}

\def\eqm{\mathbin{\equiv}}                     
\def\noeqm{\mathbin{\not\!\equiv}}  
%\newcommand{\flam}[2]{\lambda_{#1}\; .\; #2}
\def\existential#1#2{\exists_{#1}\;.\; #2}
\def\existencial#1#2{\exists_{#1}\;.\; #2}

\def\pv#1#2{\langle #1 \rangle #2}
\def\nc#1#2{[#1]#2}
\def\pvo#1#2{\langle \! \! \! \langle #1 \rangle \! \! \! \rangle\, #2}
\def\nco#1#2{\llbracket #1 \rrbracket #2}
\def\cvg#1{\llbracket \downarrow \rrbracket #1}
\def\cvgr#1#2{\llbracket #1 \downarrow \rrbracket #2}
\def\cvgl#1#2{\llbracket \downarrow  #1 \rrbracket #2}
\def\cvglr#1#2{\llbracket \downarrow  #1 \downarrow \rrbracket #2}
\def\lfp#1#2{\mu {#1}\, .\, {#2}}
\def\lpf#1#2{\mu {#1}\, .\, {#2}}
\def\gfp#1#2{\nu {#1}\, .\, {#2}}
\def\gpf#1#2{\nu {#1}\, .\, {#2}}
\def\mset#1{\vvv #1 \vvv}
\def\vvv{\vert \! \vert}
\def\mnc#1{\vvv [#1] \vvv}
\def\mpv#1{\vvv \langle #1 \rangle \vvv}
\def\bcomp#1{#1^{\text{c}}}
\def\eqm{\mathbin{\simeq}}
\def\noeqm{\mathbin{\not\!\simeq}}
\def\universal#1#2{\forall_{#1}\;.\; #2}
\def\existential#1#2{\exists_{#1}\;.\; #2}
\def\oexistential#1#2{\exists^{1}_{#1}\;.\; #2}
\def\MM{\mathcal{M}}
\def\uppaal{\textsc{Uppaal}}
\def\cc#1{\mathcal{C}(#1)}
\def\R{\mathcal{R}}
\def\TL#1{\mathcal{T}(#1)}
\def\ET#1{\mathsf{ExecTime(#1)}}



\begin{document}

\title{Verification of Timed Systems}

\frame[plain]{\titlepage}

%----------------------------------------------------------------------------------
\begin{slide}{The Satisfaction Problem}

        Given a system $S$ and a property $\varphi$ show
        that
        \[
                \alert { \tikzmark{x1} \llbracket {S} \rrbracket\, \tikzmark{x2} 
                \qquad
        \tikzmark{y1} \models \tikzmark{y2} \qquad \varphi }
        \]


        \begin{tikzpicture}[overlay,remember picture, box/.style
        = {rounded corners}, pin edge={-Stealth,thick, red}] 
        \coordinate (x1) at
        ($({pic cs:x1})+(-0ex, 1.5ex)$); \coordinate (x2) 
                at ($({pic cs:x2})+(-1.5ex,-0.5ex)$); 
        \node[semitransparent, fit=(x1) (x2),
        pin=below:\tiny{semantics of $S$}]
        {};
        \end{tikzpicture}

        \begin{tikzpicture}[overlay,remember picture, box/.style
        = {rounded corners}, pin edge={-Stealth,thick, red}] 
        \coordinate (y1) at
        ($({pic cs:y1})+(0.5ex, 1.5ex)$); 
        \coordinate (y2) 
                at ($({pic cs:y2})+(-1.5ex,-0.5ex)$); 
        \node[semitransparent, fit=(y1) (y2),
                pin=below:\tiny{$\hspace{2.5cm}
                        \models :: \text{Models} \times \text{Formulae}
                \to \{ \mathtt{tt}, \mathtt{ff} \}$}]
        {};
        \end{tikzpicture}

        The choice of which \alert{\underline{logical language}} to use for
        writing $\varphi$ depends on the underlying computational paradigm
\end{slide}

\begin{slide}{A Logical Language for Timed Systems}

	Variant of Computation Tree Logic with two types of formulae
        \begin{center}
                \fbox{ 
                        description of \alert{\underline{state}} and 
                        \alert{\underline{path}} properties 
                }
        \end{center}
\end{slide}

\begin{frame}{State Formulae}

        \begin{block}{Grammar}
        $
        \Psi\; ::=\; \ell ~|~ c ~|~ \texttt{deadlock} ~|~ \texttt{not~}\Psi ~|~ \Psi
        \texttt{ or }\Psi ~|~ \Psi \texttt{ and }\Psi 
        $
        \end{block}

        We can thus express \alert{\underline{current locations}} $\ell$, 
        \alert{\underline{clock constraints}} $c \in \mathcal{C}(C)$, 
        and the presence of \alert{\underline{deadlocks}}
\end{frame}

\begin{slide}{Back to the Annoying Lamp}

  \centering
  \includegraphics[scale=0.30]{./images/Lamp}

  \begin{block}{Exercise}
          Write formulae for the following statements
  \begin{enumerate}
    \item The lamp is on \texttt{low} mode
    \item Not \texttt{off} and $y>25$
    \item If it is \texttt{low} or \texttt{bright} then $y\leq 3600$
  \end{enumerate}
  \end{block} 
\end{slide}

\begin{slide}{Path Formulae}
\newcommand{\Boxc}{\dgold{\Box}}
\newcommand{\Diamondc}{\dgold{\Diamond}}
\newcommand{\Ac}{\dkb{A}}
\newcommand{\Ec}{\dkb{E}}

\begin{block}{Grammar}
$\Pi\; ::=\;  \Ac \Boxc\, \Psi\, \mid\, \Ac\Diamondc\, \Psi\, \mid\, \Ec \Boxc\, \Psi\, \mid\, \Ec \Diamondc\, \Psi\, \mid\,  \Phi\, \leadsto\, \Psi
$
\end{block}

\dkb{$A$, $E$} quantify (universally and existentially) over paths

\dgold{$\Box$, $\Diamond$} quantify (universally and existentially) over
states in a path

\bigskip
Paths are seen as possible system executions
\end{slide}

\begin{slide}{Path Formulae in Pictures}
\centering

~\\
\begin{tabular}{cc}
  \Large $A \Box\, \varphi$ & \Large $A \Diamond \, \varphi$ \\
 \includegraphics[width=2.8cm]{./images/AA.jpg} &
 \hspace{1cm} \includegraphics[width=2.8cm]{./images/AE.jpg}
\end{tabular}

~\\[2mm]

% \begin{block}{$E \Box\, \varphi$ and $E \Diamond\, \varphi$}
\begin{tabular}{cc}
  \Large $E \Box\, \varphi$ & \Large $E \Diamond\, \varphi$ \\
 \includegraphics[width=2.8cm]{./images/EA.jpg} &   \hspace{1cm} \includegraphics[width=2.8cm]{./images/EE.jpg}
\end{tabular}
% \end{block}
\end{slide}

\begin{slide}{Safety Properties}

        \begin{block}{$A \Box\, \varphi \hspace{0.5cm} E \Box\, \varphi$}
        Something bad will \alert{\underline{never}} happen, \eg\ 

\begin{itemize}
\item Temperature will never exceed the prescribed threshold
\item System never reaches a deadlock
\item At least one execution in which it never reaches a deadlock
\end{itemize}

\end{block}

\end{slide}

\begin{slide}{Reachability Properties}

\begin{block}{$E \Diamond\, \varphi$}

        Something good \alert{\underline{can}} happen, \eg\

	\begin{itemize}
                \item All adventurers reach the other side
                \item All adventurers reach the other side in $\leq$ 17 minutes
	\end{itemize}

\end{block}

\end{slide}


\begin{slide}{Path Formulae}

        For all paths if $\varphi$ holds at some point then $\psi$ will also
        hold later on

\small \centering

% \begin{block}{$\varphi\, \leadsto\, \psi$}
\begin{center}
        \Large $\varphi\, \leadsto\, \psi$ \\[2mm]
 \includegraphics[width=5cm]{./images/LeadsTo.jpg} 
 \end{center}
% \end{block}

\end{slide}


%----------------------------------------------------------------------------------
\begin{slide}{Liveness Properties}

\begin{block}{$A\, \Diamond\, \phi \hspace{0.5cm} \phi\, \leadsto \, \psi$}
        If something happens then something \alert{\underline{good}}
        \alert{\underline{eventually}} happens
\begin{itemize}
	\item When pressing ON the
		television will eventually turn on
        \item If the philosopher requests a fork she will 
                get it
        \item If the plane asks to land it will eventually land
\end{itemize}

\end{block}

\end{slide}

\begin{frame}{Exercises}
	Write the sentences below in CTL
	\begin{enumerate}
		\item The system never enters in deadlock
		\item The location $\ell$ is reachable
		\item In all executions we reach location $\ell$
		\item If we reach location $\ell$ we will inevitably reach
			location $s$
		\item There exists at least one execution where variable
			\texttt{i} is always below or equal \texttt{10}
		\item The two philosophers never eat at the same time
	\end{enumerate}
\end{frame}

\begin{frame}{Back to the Annoying Lamp}

        \centering
        \includegraphics[scale=0.25]{./images/Lamp.pdf}

        \begin{block}{Exercise}
	\begin{enumerate}
                \item The lamp can become bright;
                \item The lamp will eventually become bright;
                \item The lamp can never be on for more than 3600s;
                \item It is possible to never turn on the lamp;
                \item Whenever the light is bright, the clock $y$
                        is non-zero;
                \item Whenever the light is bright it will eventually
                        become off.
	\end{enumerate}
        \end{block}
\end{frame}

\section{Chapter's Conclusion}

\begin{frame}{Key Takeaways}

        Communicating systems as \tikzmark{z1} \alert{\underline{timed automata}} 
        \tikzmark{z2}

        \begin{tikzpicture}[overlay,remember picture, box/.style
        = {rounded corners}, pin edge={-Stealth,thick, red}] 
        \coordinate (z1) at
        ($({pic cs:z1})+(0.5ex, 1.5ex)$); 
        \coordinate (z2) 
                at ($({pic cs:z2})+(-1.5ex,-0.5ex)$); 
        \node[semitransparent, fit=(z1) (z2),
                pin=below:\tiny{Syntax}]
        {};
        \end{tikzpicture}

        \vspace{0.5cm}
        \alert{\underline{Semantics}} for rigorous analysis and verification

        \vspace{1cm}
        \alert{\underline{UPPAAL}} as an important tool of the cyber-physical engineer
\end{frame}

\begin{frame}{Key Takeaways}


        \vspace{0.4cm}
        \alert{\underline{Time}} as the main physical process \dots

        \dots\ others will appear in the next lectures

        \vspace{0.4cm}
        Barely scratched the surface \dots

        \dots\ more about the theory of timed automata 
        in the website
\end{frame}
\begin{frame}{Open Challenges}
        \alert{\underline{Quantum}} communicating systems and computational
        tools

        Reasoning precisely about \alert{\underline{imprecisions}}

        The thorny
        \href{https://www.youtube.com/watch?v=IzSs_gJDVzI}{\alert{\underline{Reachability
        Problem}}}

        \dots
\end{frame}
\end{document}
